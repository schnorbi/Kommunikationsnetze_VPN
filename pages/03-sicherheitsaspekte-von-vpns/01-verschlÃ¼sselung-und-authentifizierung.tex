\section{Verschlüsselung und Authentifizierung}

Ein zentraler Teil eines \gl{VPN} besteht in der Verschlüsselung und Authentifizierung. Der Datenverkehr soll nicht nur verschlüsselt sein, sondern die Einwahl in das \gl{VPN} soll auch nur befugten Teilnehmern gestattet werden.

Die Verschlüsselung des Datenverkehrs sorgt dafür, dass die Daten, die zwischen dem Benutzer und dem \gl{VPN}-Server übertragen werden, nicht von Dritten gelesen oder modifiziert werden können. Dies verringert den Angriffsvektor auf die Kommunikation zwischen Benutzer und \gl{VPN}-Server enorm und kann auch in gewissen Umständen den Zugriff auf ungesicherte oder unsichere Anwendungen ermöglichen. Moderne \gl{VPN}s verwenden häufig starke Verschlüsselungsprotokolle wie \gls{AES} mit 256-Bit-Schlüssellänge, das als praktisch unknackbar gilt \cite{BSI_Recommendations_and_key_lengths}.

Bei der Authentifizierung wird die Einwahl auf den \gl{VPN}-Server kontrolliert. Je nach Implementierung der Software für die Kommunikation, können Faktoren wie Schlüssel, digitale Zertifikate und Zwei-Faktor-Authentifizierung gefordert werden, bevor es zu einem erfolgreichen Verbindungsaufbau kommt. Zusätzlich können bestehende Verbindungen jederzeit untersagt werden, um beispielsweise den weiteren Zugriff eines kompromittierten Benutzers zu untersagen. 

Zusammen bieten Verschlüsselung und Authentifizierung einen robusten Schutz gegen verschiedene Bedrohungen wie Man-in-the-Middle-Angriffe und unbefugten Zugriff, indem sie sicherstellen, dass die Daten sowohl vertraulich als auch integer bleiben.