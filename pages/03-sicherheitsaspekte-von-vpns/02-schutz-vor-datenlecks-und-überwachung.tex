\section{Schutz vor Datenlecks und Überwachung}

Ein weiterer wichtiger Sicherheitsaspekt von \gl{VPN}s ist der Schutz vor Datenlecks und Überwachung. Durch eine fehlerhafte Konfiguration oder Implementation, können diese jederzeit auftreten, ohne dass der Benutzer etwas davon mitbekommt.

Datenlecks können beispielsweise durch eine Fehlkonfiguration oder verlorene Verbindung zum \gl{VPN}-Server passieren. Hier kommt es zu Datenverkehr außerhalb der definierten sicheren Verbindung und ein Angreifer kann durch einen Man-in-the-Middle-Angriff die Daten mitschneiden. Viele Anwendungen bieten hier eine sogenannte Kill-Switch-Funktionalität, welche Datenverkehr außerhalb der Verbindung zum \gl{VPN}-Server untersagt \cite{OpenVPN_KillSwitch, NordVPN_KillSwitch}. Das Problem bei dieser Art an Datenlecks, ist die fehlende Transparenz gegnüber des Benutzers. 

Um der Überwachung im Netzwerk zu entgehen, wie beispielsweise Trackern, öffentlichen Netzwerken oder Internetanbietern, werden \gl{VPN}s auch gerne eingesetzt. Der Datenverkehr wird hier verschlüsselt an allen Teilnehmern zwischen dem Benutzer und dem \gl{VPN}-Server vorbeigetragen und erst über den \gl{VPN}-Server ins Internet kommuniziert. Dadurch können sich beispielsweise alle Teilnehmer auf dem \gl{VPN}-Server eine IP-Adresse teilen und dadurch einen anderen Standort vortäuschen oder auch lokale Zensuren umgegangen werden, dies erschwert die Überwachung und Einschränkung von Online-Aktivitäten enorm.