\chapter{Einleitung} \label{introduction}
In einer zunehmend vernetzten Welt, in der Daten zu einem der wertvollsten Güter geworden sind, gewinnt die Sicherheit dieser Daten immens an Bedeutung. In Deutschland nimmt die Cyberkriminalität jedes Jahr zu. So hat sich die Anzahl der Cyberkriminalitätsfälle, welche durch die deutsche Polizei verzeichnet wurden, zwischen 2012 und 2022 mehr als verdoppelt. \cite{Filed_cybercrime_cases_Germany_2022} Ebenso erschreckend sind die Verluste durch Cyberkriminalität. So entstand allein deutschen Unternehmen im Jahr 2023 ein Schaden in Höhe von 205,9 Milliarden Euro. \cite{Cybercrime_financial_loss_in_Germany_2023} Um dies zu verhindern, erfordert die fortschreitende Digitalisierung und Globalisierung Lösungen, die es ermöglichen, weltweit auf Informationen zuzugreifen und dabei die größtmögliche Sicherheit zu gewährleisten. Eine dieser Lösungen ist das \gls{VPN}, also ein nicht-öffentliches Netzwerk, welche nur virtuell besteht. 

Durch die Verschlüsselung des Datenverkehrs und die Einrichtung eines sicheren Tunnels zwischen dem Endgerät des Nutzers und dem Zielnetzwerk ermöglichen \gls{VPN}s lokale Sicherheit, indem sie sensible Informationen vor unbefugtem Zugriff schützen und die Integrität der Datenübertragung gewährleisten. Aber \gls{VPN}s bieten mehr als nur Sicherheit; sie ermöglichen es Nutzern auch, geografische Beschränkungen zu umgehen und auf Inhalte zuzugreifen, die in ihrem Land möglicherweise blockiert sind. Daher haben \gls{VPN}s eine entscheidende Doppelrolle, indem sie sowohl globale Konnektivität als auch lokale Sicherheit ermöglichen. \cite{Vergleich_der_besten_VPN-Protokolle_Nord_VPN}

Das Grundprinzip eines \gls{VPN}s besteht darin, dass eine sichere und verschlüsselte virtuelle Verbindung, auch Tunnel genannt, zwischen dem Endgeräte eines Nutzers und einem entfernten Server hergestellt wird. Die Endgeräte kommunizieren dabei anderes als im Heimnetzwerk nicht über eine physische Verbindung oder einen zentralen Router, sondern über das ungeschützte öffentliche Internet. Sobald die Verbindung aufgebaut ist, werden alle Daten, die zwischen dem Gerät und dem Server übertragen werden, durch sichere Verschlüsselungsalgorithmen geschützt. Dies stellt sicher, dass die Daten vor unbefugtem Zugriff und Abhörversuchen durch Dritte, wie Hacker oder Internetprovider, geschützt sind. \cite{Vergleich_der_besten_VPN-Protokolle_Nord_VPN, Wie_funktioniert_ein_Virtual_Private_Network_VPN}

Der verschlüsselte Tunnel leitet den gesamten Internetverkehr des Nutzers durch den \gls{VPN}-Server, wodurch die ursprüngliche IP-Adresse des Nutzers maskiert wird und die des \gls{VPN}-Servers übernommen wird. Dies ermöglicht nicht nur den Schutz der Privatsphäre, sondern auch den Zugriff auf Inhalte, die zum Beispiel geografisch eingeschränkt sein könnten, da der Nutzer virtuell an dem Standort des \gls{VPN}-Servers erscheint. \cite{Vergleich_der_besten_VPN-Protokolle_Nord_VPN, Wie_funktioniert_ein_Virtual_Private_Network_VPN}

Die Authentifizierung ist ein weiterer wichtiger Bestandteil eines \gls{VPN}s. Dazu erfolgt zum Beispiel ein Schlüsselaustausch beim Verbindungsaufbau. So kann der Server sicherstellen, dass nur autorisierte Geräte auf das virtuelle Netzwerk zugreifen können. Verschiedene Protokolle wie OpenVPN, \gls{IPSec} und WireGuard bieten unterschiedliche Authentifizierungsansätze, wobei jedes Protokoll spezifische Stärken in Bezug auf Sicherheit, Geschwindigkeit und Benutzerfreundlichkeit aufweist. Insgesamt bietet ein \gls{VPN} aber eine robuste Lösung für den Schutz der Privatsphäre und die Sicherstellung der Datenintegrität in der digitalen Kommunikation. \cite{Wie_funktioniert_ein_Virtual_Private_Network_VPN}

\gls{VPN}s bieten vielseitige Anwendungsmöglichkeiten im beruflichen und privaten Bereich. Im Berufsleben ermöglichen sie eine sichere Anbindung von Home-Office-Arbeitsplätzen und mobilen Zugriff für Außendienstmitarbeiter auf zentrale Unternehmensanwendungen. Sie verbinden räumlich getrennte Standortnetze, was für Unternehmen, Universitäten und staatliche Einrichtungen von Bedeutung ist. Öffentliche WLAN-Hotspots sind mit \gls{VPN}s sicherer, da die Verschlüsselung unbefugten Zugriff und Ausspähversuche verhindert. Auch der Zugriff auf das Heimnetzwerk über WLAN-Hotspots wird durch \gls{VPN}s geschützt. Für Smart-Home-Geräte bieten \gls{VPN}s eine sichere Alternative zur Cloud-Steuerung, da sie die direkte Steuerung über den Heimnetz-Router ermöglichen und somit das Risiko von Zugriffsversuche durch Dritte vermieden wird. Außerdem bieten \gls{VPN}s die Möglichkeit auf heimische Inhalte während eines Auslandsaufenthalts zuzugreifen, indem sie dem Nutzer eine \gls{IP}-Adresse des Heimatlandes zuweisen und so eventuell Geo-Blocker umgehen. \cite{Wie_funktioniert_ein_Virtual_Private_Network_VPN}

Angesichts der wachsenden Bedrohung durch Cyberkriminalität und der ständigen Notwendigkeit, sensible Daten zu schützen, ist es entscheidend, den Mehrwert moderner \gls{VPN}-Lösungen sowohl für Unternehmen als auch für Privatanwender zu verstehen. Diese Hausarbeit wird daher die verschiedenen Aspekte von \gls{VPN}-Lösungen untersuchen und ihre Bedeutung für die Sicherheit und Privatsphäre in der heutigen digitalen Welt beleuchten.