\section{Smart Home - \gls{VPN} ins Heimnetz} \label{vpn-home}
Ein \gls{VPN} bietet im Kontext eines Smart Homes zahlreiche Vorteile, insbesondere in Bezug auf Sicherheit und Zugänglichkeit. Durch den Einsatz eines \gls{VPN}s im eigenen Heimnetzwerk kann die Notwendigkeit einer direkten Freigabe von Smart-Home-Geräten ins Internet vollständig vermieden werden. Dies reduziert das Risiko von Cyberangriffen erheblich, da die Geräte nicht mehr direkt aus dem Internet erreichbar sind und somit eine zusätzliche Sicherheitsschicht entsteht.  Der Zugriff auf die Smart Home Geräte erfolgt stattdessen über das \gls{VPN}, das eine verschlüsselte und sichere Verbindung herstellt. \cite{jeliskoski2018securing, Klein-Hennig2017-ia}

Ein weiterer wesentlicher Vorteil eines \gls{VPN}s im Smart Home ist der vereinfachte Zugang unabhängig vom Standort. Mit einem \gls{VPN} können Nutzer von überall auf der Welt auf ihr Heimnetzwerk zugreifen, als wären sie physisch zu Hause. Dies bedeutet, dass alle smarten Geräte, von Überwachungskameras über Thermostate bis hin zu Beleuchtungssystemen, sicher und bequem verwaltet werden können. Der Benutzer verbindet sich einfach mit dem VPN-Server im eigenen Heimnetzwerk und erhält dadurch Zugriff auf alle vernetzten Geräte. Diese Art der Verbindung ist nicht nur sicher, sondern auch einfach zu handhaben und bietet eine hohe Flexibilität. \cite{Klein-Hennig2017-ia}

Darüber hinaus schützt ein \gls{VPN} die Verwaltung der Smart Home Geräte. Administratorzugriffe und sensible Daten werden durch die verschlüsselte Verbindung vor unbefugtem Zugriff geschützt. Die Verwendung eines \gls{VPN}s stellt sicher, dass nur autorisierte Benutzer, die über die richtigen Zugangsdaten verfügen, in der Lage sind, Einstellungen zu ändern oder auf die Steuerungsfunktionen der Smart Home Geräte zuzugreifen. \cite{jeliskoski2018securing, Klein-Hennig2017-ia} 

Zusammengefasst bietet der Einsatz eines \gls{VPN}s im Smart Home erhebliche Sicherheitsvorteile und erleichtert den Zugriff auf vernetzte Geräte. Es macht eine direkte Freigabe ins Internet überflüssig und schützt die Verwaltung der Geräte durch sichere, verschlüsselte Verbindungen. Gleichzeitig ermöglicht es den Nutzern einen einfachen und sicheren Zugriff auf ihr Heimnetzwerk, unabhängig davon, wo sie sich gerade befinden.