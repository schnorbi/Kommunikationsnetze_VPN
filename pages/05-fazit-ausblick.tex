\chapter{Fazit \& Ausblick} \label{fazit}

In dieser Hausarbeit wurde die Bedeutung von \gls{VPN}s in der modernen digitalen Welt ausführlich untersucht. \gls{VPN}s sind essenzielle Werkzeuge, die sowohl Sicherheit als auch Datenschutz im Internet gewährleisten. Durch die Verschlüsselung des Datenverkehrs und die Authentifizierung der Benutzer bieten \gls{VPN}s robusten Schutz vor Cyberangriffen und unbefugtem Zugriff. Sie ermöglichen es Nutzern, geografische Beschränkungen zu umgehen und auf gesperrte Inhalte zuzugreifen, was besonders in Ländern mit restriktiven Internetgesetzen von großer Bedeutung ist. Insgesamt tragen \gls{VPN}s dazu bei, die digitale Kommunikation sicherer und flexibler zu gestalten, indem sie eine sichere Verbindung über unsichere Netzwerke ermöglichen.

Bei dem Vergleich von den verschiedenen \gls{VPN}-Protokollen, fällt besonders die unterschiedliche Komplexität auf. \gls{IPSec} besitzt beispielsweise mehrere Modi mit unterschiedlichen Funktionsweisen und mehrere unterliegende Protokolle um einen erfolgreichen Datenverkehr zu sichern. Bei OpenVPN wird explizit zwischen Server und Client unterschieden, was zu einem erhöhten Verwaltungsaufwand führt. WireGuard auf der anderen Seite, eine vergleichsweise neue Technologien, verzichtet auf diese komplexen Eigenschaften. Dadurch erfüllt WireGuard seine eigenen Ansprüche: eine einfachere Anwendung durch ein übersichtliches Protokoll und Konfiguration, sicherer Datenverkehr durch moderne Kryptografie und einen schnelleren Datenverkehr durch eine Kombination aus optimierten Implementierungen und modernen Algorithmen.

Die zukünftige Entwicklung von \gls{VPN}-Technologien wird sich voraussichtlich auf die Integration mit anderen innovativen Technologien konzentrieren, wie beispielsweise \gls{SDN}s. Die Kombination von \gls{VPN}s und \gls{SDN} bietet zahlreiche Vorteile, darunter die zentrale Verwaltung und Kontrolle von \gls{VPN}-Verbindungen sowie die Automatisierung und Orchestrierung von Netzwerkaufgaben. Dies kann zu einer effizienteren und sichereren Netzwerkverwaltung führen. Ein wichtiges Thema was diese Art der Anwendung zudem voran treibt ist die steigende Interesse nach Zero-Trust Lösungen innerhalb von Unternehmen, um die eigene IT-Infrastruktur besser abzusichern \cite{Statista_ZeroTrust}. 

Darüber hinaus wird die fortschreitende Digitalisierung und der zunehmende Einsatz von \gls{IoT}-Geräten die Bedeutung von \gls{VPN}s weiter erhöhen. In diesem Zusammenhang wird die Sicherstellung der sicheren Kommunikation und Verwaltung von \gls{IoT}-Geräten eine zentrale Rolle spielen.

Ein weiterer wichtiger Trend ist die Entwicklung von noch leistungsfähigeren Verschlüsselungs- und Authentifizierungsmethoden, um den wachsenden Bedrohungen im Cyberraum wirksam zu begegnen. Es bleibt abzuwarten, wie sich \gls{VPN}s weiterentwickeln werden, um den zukünftigen Anforderungen an Sicherheit und Datenschutz gerecht zu werden.